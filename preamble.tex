% Используем XeLaTeX – настраиваем шрифт через fontspec
\usepackage{fontspec}
\setmainfont{Times New Roman}

% Настройка полей (левое: 30 мм, правое: 15 мм (можно 10 мм), верхнее: 20 мм, нижнее: 20 мм)
\usepackage[a4paper,top=20mm,bottom=20mm,left=30mm,right=15mm]{geometry}

% Межстрочный интервал 1.5 (14 пт рекомендуемый размер шрифта, можно задавать через документ или в настройках Overleaf)
\usepackage{setspace}
\onehalfspacing

% Абзацный отступ 1.25 см
\setlength{\parindent}{1.25cm}
\setlength{\parskip}{0pt}

% Нумерация страниц: арабскими цифрами, по центру нижней части
\usepackage{fancyhdr}
\pagestyle{plain}
\fancyhf{}
\cfoot{\thepage}

% Настройка заголовков через titlesec
\usepackage{titlesec}
% Заголовки глав: прописными буквами, полужирно, по центру, без точки в конце
\titleformat{\chapter}[hang]{\bfseries\Large\centering}{\thechapter}{1em}{}
% Заголовки разделов и подразделов – с абзацным отступом
\titleformat{\section}[hang]{\bfseries\large}{\thesection}{1em}{}
\titleformat{\subsection}[hang]{\bfseries}{\thesubsection}{1em}{}
\titlespacing{\chapter}{0pt}{0pt}{20pt}
\titlespacing{\section}{0pt}{12pt}{6pt}
\titlespacing{\subsection}{0pt}{6pt}{3pt}

% Настройка содержания: точечные лидеры между заголовками и номерами страниц
\usepackage{tocloft}
\renewcommand{\cftdotsep}{1}
\renewcommand{\cftchapleader}{\cftdotfill{\cftdotsep}}

% Подключаем отдельные конфигурационные файлы для списков, формул, таблиц и изображений
\usepackage{enumitem}
\setlist{nosep} % Убирает дополнительные отступы между элементами списков
\setlist[itemize]{leftmargin=*, label=\textbullet}
\setlist[enumerate]{leftmargin=*}

\usepackage{amsmath}
\numberwithin{equation}{chapter} % Формулы нумеруются по главам (например, (1.1), (1.2) и т.д.)
% Подключаем нужные пакеты для таблиц и оформления
\usepackage{booktabs}   % Для красивых горизонтальных линий (\toprule, \midrule, \bottomrule)
\usepackage{multirow}   % Для объединения ячеек по вертикали
\usepackage{caption}    % Для \ContinuedFloat (продолжение таблицы)

% ------------------------
% Шаблон: Многоуровневый заголовок
% ------------------------
\newcommand{\ExampleMultiLevelTable}{%
\begin{table}[h!]
\caption{Пример построения таблицы}
\centering
\begin{tabular}{|c|c|c|c|}
\hline
\multirow{2}{*}{\textbf{Заголовок}} & \multicolumn{2}{c|}{\textbf{Заголовок колонки}} & \multirow{2}{*}{\textbf{Заголовок колонки}}\\
\cline{2-3}
& \textbf{Подзаголовок} & \textbf{Подзаголовок} & \\
\hline
A & B & C & D \\
\hline
\end{tabular}
\end{table}
}

% Шаблон: "Разбитая" таблица (две части под одним номером)
\newcommand{\ExampleSplitTable}{%
  % --- Первая часть ---
  \begin{table}[h!]
    \LeftCaption
    \caption{Пример переноса таблицы}
    \centering
    \begin{tabular}{|c|c|c|c|}
    \hline
    \textbf{Заголовок} & \textbf{Заголовок колонки} & \textbf{Заголовок колонки} & \textbf{Заголовок колонки} \\
    \hline
    1 & 2 & 3 & 4 \\
    \hline
    2 & 3 & 4 & 5 \\
    \hline
    \end{tabular}
  \end{table}

  % --- Продолжение ---
  \begin{table}[h!]
    \ContinuedFloat
    \RightCaption
    \caption*{Продолжение таблицы 5.1}
    \centering
    \begin{tabular}{|c|c|c|c|}
    \hline
    \textbf{Заголовок} & \textbf{Заголовок колонки} & \textbf{Заголовок колонки} & \textbf{Заголовок колонки} \\
    \hline
    3 & 4 & 5 & 6 \\
    \hline
    4 & 5 & 6 & 7 \\
    \hline
    \end{tabular}
  \end{table}
}

\usepackage{caption}
\usepackage{longtable}
\usepackage{etoolbox}
\DeclareCaptionLabelSeparator{emdash}{\space---\space}  % пробел, длинное тире, пробел

% Настройка выравнивания заголовков
\captionsetup[table]{position=t, singlelinecheck=false, justification=raggedright, labelsep=emdash, name=Таблица}

% Переключатель для смены выравнивания заголовков
\newcommand*\RightCaption{\captionsetup{justification=raggedleft}}
\newcommand*\LeftCaption{\captionsetup{justification=raggedright}}


\usepackage{graphicx}
\graphicspath{{images/}} % Папка с изображениями
\usepackage{caption}
\DeclareCaptionLabelSeparator{emdash}{\space---\space}  % пробел, длинное тире, пробел

\captionsetup[figure]{position=b, singlelinecheck=false, labelsep=emdash, justification=centering, name=Рисунок}

% % Adjust figure numbering for normal sections
% \renewcommand{\thefigure}{\thesection.\arabic{figure}}
%
% % Adjust figure numbering for appendices: Use letter-based section numbering (A, B, ...)
% \newcommand{\appendixfigure}{\renewcommand{\thefigure}{\Alph{section}.\arabic{figure}}}
% \counterwithin{figure}{section}


% Дополнительные пакеты
\usepackage[backend=biber,style=gost-numeric,sorting=none]{biblatex}
\addbibresource{bibliography.bib}
\usepackage{booktabs}

% какие буквы можно использовать в TOC, тк буквы Ё, З, Й, О, Ч, Ъ, Ы, Ь не должны встречаться
\def\cyrtoc#1{\ifcase #1\or А\or Б\or В\or Г\or Д\or Е\or Ж\or И\or К\or Л\or М\or Н\or П\or Р\or С\or Т\or У\or Ф\or Х\or Ц\or Ш\or Щ\or Э\or Ю\or Я\else \@ctrerr \fi}

\newcommand\originalappendix{}
\let\originalappendix\appendix
\renewcommand{\appendix}{%
  \originalappendix
  \renewcommand{\thechapter}{\cyrtoc{\value{chapter}}}
  \renewcommand{\appendixname}{ПРИЛОЖЕНИЕ}
  \titleformat{\chapter}{\Large\bfseries\centering}{\MakeUppercase{\chaptertitlename}.\ \thechapter}{4pt}{}

  \addtocontents{toc}{\protect\value{tocdepth}=\chaptertocdepth}
  \renewcommand\addchaptertocentry[2]{%
    \addtocentrydefault{chapter}{}{\appendixname~##1 ##2}
  }
}

