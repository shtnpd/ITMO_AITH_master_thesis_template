% Подключение пакета для теорем
\usepackage{amsthm}
\usepackage{amsmath}

% Локализация заголовков теорем на русский
\newtheoremstyle{russian}% имя
  {3pt}% Отступ сверху
  {3pt}% Отступ снизу
  {\itshape}% Шрифт тела теоремы
  {}% Отступ слева
  {\bfseries}% Шрифт заголовка
  {}% Знак после заголовка
  { }% Интервал после заголовка
  {\thmname{#1}~\thmnumber{#2}\thmnote{~(#3)}}% Формат

\theoremstyle{russian}

% Определяем теоремы с нумерацией по главам
\newtheorem{theorem}{Теорема}[chapter]
\newtheorem{lemma}[theorem]{Лемма}
\newtheorem{proposition}[theorem]{Утверждение}
\newtheorem{corollary}[theorem]{Следствие}

% Ненумерованное определение (например, доказательство)
\theoremstyle{definition}
\newtheorem*{definition}{Определение}
\newtheorem*{remark}{Замечание}

% Доказательство (встроено в amsthm)
\renewcommand{\proofname}{Доказательство}
\numberwithin{equation}{chapter} % Формулы нумеруются по главам (например, (1.1), (1.2) и т.д.)