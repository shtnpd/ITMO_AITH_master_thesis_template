\chapter{АНАЛИЗ ЛИТЕРАТУРЫ}

В данном разделе рассматриваются основные методы и подходы, применяемые для решения задач классификации данных с использованием методов машинного обучения и глубокого обучения.

Например, Иванов и Петров (2019) предлагают использовать ансамблевые методы для повышения точности классификации, в то время как Smith и Johnson (2020) демонстрируют эффективность применения глубоких нейронных сетей. Один из известных подходов описывается через формулу взаимосвязи массы и энергии:
\begin{equation}
E = mc^2,
\end{equation}
где \(E\) --- энергия, \(m\) --- масса, \(c\) --- скорость света.

Другой важный аспект исследований --- функция потерь, которую можно представить в виде:
\begin{equation}
L(y, \hat{y}) = \frac{1}{n} \sum_{i=1}^{n} (y_i - \hat{y}_i)^2,
\end{equation}
где \(y_i\) --- истинное значение, \(\hat{y}_i\) --- предсказанное значение, \(n\) --- количество наблюдений.

С помощью формулы (1.2) можно получить значение функции потерь, чтобы проанализировать качество предсказания. 

Анализ литературы показывает, что сочетание нескольких методов может привести к значительному улучшению качества классификации. Таким образом, обзор источников подтверждает актуальность выбранной темы и обосновывает необходимость дальнейших исследований в данной области.

